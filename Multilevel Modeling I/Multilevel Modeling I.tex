\documentclass{beamer}
\setbeamertemplate{caption}[numbered]
\usepackage{graphicx}
\usepackage{svg}
\usepackage[utf8]{inputenc}
\usepackage[english]{babel}
\graphicspath{{Graphs/}}
\usepackage{comment}
\usepackage{verbatim}
\usepackage{hyperref}
\mode<presentation>
{
	\usetheme{AnnArbor}
	\usecolortheme{crane}
}

\title[MLM I]{Multilevel Modeling (Part 1)}
\subtitle[ISRC Workshop]{Basic Random-Intercept and Random-Slope Modeling}
\author[Wallace]{Desmond D. Wallace}
\institute[University of Iowa]{Department of Political Science\\The University of Iowa\\Iowa City, IA}

\date{March 23, 2018}

\begin{document}

\begin{frame}
	\titlepage
\end{frame}

%\begin{frame}
%	\frametitle{Table of Contents}
%	\tableofcontents
%\end{frame}

\section{REVIEW: Regression Basics}

\begin{frame}
	\frametitle{Regression Highlights}
	\begin{itemize}
		\item A way to summarize the relationship between variables.
		\item Assuming there is a relationship between Y and the independent variable(s).
		\item Relationship may be linear (OLS) or non-linear (CLDV).
		\item Regression helps our understanding of how our dependent variable of interest changes when one or more independent variables vary, while holding remaining variables fixed.
		\item PRF: $y_{i}=\beta_{0}+\beta_{1}x_{1}+\cdots+\beta_{k}x_{k}+\varepsilon_{i}$
		\item SRF: $y_{i}=b_{0}+b_{1}x_{1}+\cdots+b_{k}x_{k}+e_{i}$
	\end{itemize}
\end{frame}

\subsection{Regression Assumptions}

\begin{frame}
	\frametitle{Model Specification}
		\begin{enumerate}
			\item Linearity in the parameters
			\item The number of observations n must be greater than the number of
			parameters to be estimated
			\item The regression model is correctly specified
		\end{enumerate}
\end{frame}

\begin{frame}
	\frametitle{Independent Variable(s)}
		\begin{enumerate}
			\item X values are fixed in repeated sampling
			\item Variability in X values
			\item There is no perfect multicollinearity
		\end{enumerate}
\end{frame}

\begin{frame}
	\frametitle{Error Term}
		\begin{enumerate}
			\item Zero mean value of error $(e_i)$
			\item \textbf{Homoscedasticity or equal variance} of $e_i$
			\item No autocorrelation between the errors
			\item Zero covariance between $e_i$ and $X_i$
		\end{enumerate}
\end{frame}

\begin{frame}
	\frametitle{Additional Assumptions}
		\begin{enumerate}
			\item Errors are normally distributed
			\item \textbf{Errors for any two observations are independent of one another}
		\end{enumerate}
\end{frame}

\section{Multilevel Modeling}

\subsection{Violating OLS Assumptions}

\begin{frame}
	\frametitle{Multi-Stage Sampling}
		\begin{itemize}
			\item OLS assumptions imply utilization of Simple Random Sampling (SRS)
			\item However, due to cost-efficiency, multi-stage sampling approaches may be utilized instead.
			\item Researcher may randomly sample grouping units instead of individuals (cluster sampling)
			\item Examples
				\begin{itemize}
					\item Students nested in schools
					\item Respondents nested in states (countries)
					\item Patients nested in hospitals
				\end{itemize}
		\end{itemize}
\end{frame}

\begin{frame}
	\frametitle{Applying OLS to Multilevel Data}
		\begin{itemize}
			\item Biased standard errors
			\item Model Misspecification
		\end{itemize}
\end{frame}

\subsection{Introduction}

\begin{frame}
	\frametitle{New Approach}
		\begin{itemize}
			\item Best approach to analyzing nested data is a statistical approach that accounts for both within-group and between-group variation \textit{simultaneously}
			\item One approach is to conceive within-group and between-group variation as random variability
			\item One can achieve this by including \textit{random coefficient(s)} in the statistical model
		\end{itemize}
\end{frame}

\begin{frame}
	\frametitle{Multilevel Model}
		\begin{itemize}
			\item Multilevel Model (MLM) is a model where the parameters vary at more than one level
			\item Features more than one error term
			\item Variation can occur wit respect to the intercept (\textit{random intercept}) and/or the slope (\textit{random slope})
			\item This approach leads to corrected standard errors and correct model specification
		\end{itemize}
\end{frame}

\begin{frame}
	\frametitle{Multilevel Model}
		\begin{itemize}
			\item Model coefficients are now a combination of both fixed and random components
				\begin{itemize}
					\item Fixed Coefficient -- An unknown constant of nature
					\item Random Coefficient -- One which varies from sample of groups to sample of groups
				\end{itemize}
			\item Random coefficients are not estimated, and are instead predicted
			\item Instead of BLUE (Best Linear Unbiased Estimator)  coefficients, we know have BLUP (Best Linear Unbiased Prediction) coefficients
		\end{itemize}
\end{frame}

\subsection{Random Intercept}

\begin{frame}
	\frametitle{\texttt{outreg2} Example}
		%\begin{tabular}{lc} \hline
 & (1) \\
VARIABLES &  \\ \hline
 &  \\
age of respondent & 243.8*** \\
 & (38.68) \\
Is R female (yes=1 no=0)? = 1, Female & -7,629*** \\
 & (1,114) \\
Constant & 14,141*** \\
 & (1,896) \\
 &  \\
Observations & 1,201 \\
 R-squared & 0.070 \\ \hline
\multicolumn{2}{c}{ Standard errors in parentheses} \\
\multicolumn{2}{c}{ *** p$<$0.01, ** p$<$0.05, * p$<$0.1} \\
\end{tabular}

\end{frame}

\subsection{Plots}

\begin{frame}
	\frametitle{Coefficient Plots}
		\begin{itemize}
			\item Sometimes, regression models feature many variables
			\item Also, showing many numbers and stars can be difficult for some readers
			\item An alternative to reporting a table is a plot of the regression results
		\end{itemize}
\end{frame}

\begin{frame}
	\frametitle{\texttt{coefplot}}
		\begin{itemize}
			\item \texttt{coefplot} is another user-written Stata program
			\item Plots regression results in ``dot-whisker" format
				\begin{itemize}
					\item ``Dot" -- Coefficient Estimate
					\item ``Whisker" -- Confidence Interval
				\end{itemize}
			\item Basic Syntax: \texttt{coefplot}
			\item \texttt{coefplot} command is executed AFTER regression model is estimated
		\end{itemize}
\end{frame}

\begin{frame}
	\frametitle{\texttt{coefplot} Example}
		\texttt{reg realrinc age i.female}\\
		\texttt{coefplot, title("Model Results") xline(0)}
\end{frame}

\begin{frame}
	\frametitle{\texttt{coefplot} Example}
		%\includegraphics[scale=0.3]{MIVcoef01.png}
\end{frame}

\begin{frame}
	\frametitle{\texttt{coefplot} Example}
		\texttt{reg realrinc age i.female}\\
		\texttt{coefplot, title("Model Results") xline(0) drop(\_cons)}
\end{frame}

\begin{frame}
	\frametitle{\texttt{coefplot} Example}
		%\includegraphics[scale=0.3]{MIVcoef02.png}
\end{frame}

\section{Interpretation}
\subsection{Coefficients}

\begin{frame}
	\frametitle{Interpreting Coefficients}
		\begin{itemize}
			\item Can directly interpret coefficient estimates.
			\item \textit{A one unit change in $X_{k}$ leads to a $\beta_{k}$ change in $Y$ (holding all other variables constant).}
			\item Assumes $X_{k}$ is not a constituent term for an interaction variable. 
		\end{itemize}
\end{frame}

\subsection{Predicted (Fitted) Values}

\begin{frame}
	\frametitle{Predicted (Fitted) Values}
		\begin{itemize}
			\item The result of substituting values of interest for the independent variable(s).
			\item $E[Y|X]=X\hat{\beta}$
			\item Can calculate standard errors to determine if $E[Y|X=x]$ is statistically significantly different from zero.
			\item Multiple ways to calculate fitted values in Stata.
		\end{itemize}
\end{frame}

\subsection{Marginal Effects}

\begin{frame}
	\frametitle{Marginal and Discrete Change}
		\begin{itemize}
			\item Measuring the change in the dependent variable for a change in one independent variable, holding remaining independent variables constant.
				\begin{itemize}
					\item \textit{Marginal Change} is the partial derivative, or instantaneous rate of change, in the dependent variable w.r.t. an independent variable, holding remaining variables constant.
					\item \textit{Discrete Change} or \textit{First Difference} is the difference in the prediction from one specified value of an independent variable to another specified value, holding remaining variables constant.
				\end{itemize}
		\end{itemize}
\end{frame}

\begin{frame}
	\frametitle{Marginal and Discrete Change}
		\begin{itemize}
			\item Marginal Change: $\frac{\partial E[Y|X]}{\partial x_{k}}=\frac{\partial X\beta}{\partial x_{k}}=\beta_{k}$
			\item Discrete Change: $\frac{\Delta E[Y|X]}{\Delta x_{k}}=E[Y|X, x_{k}+1]-E[Y|X, x_{k}]=\beta_{k}$
		\end{itemize}
\end{frame}

\begin{frame}
	\frametitle{Marginal and Discrete Change}
		\begin{itemize}
			\item $\frac{\partial E[Y|X]}{\partial x_{k}}=\frac{\Delta E[Y|X]}{\Delta x_{k}}=\beta_{k}$, assuming there is no interaction terms.
			\item The standard error of the marginal effect is the same as the standard error of the estimated beta coefficient.
			\item \textit{For a unit increase in $x_{k}$, the expected change in $Y$ equals $\beta_{k}$, holding all other variables constant.}
			\item \textit{Having characteristic $x_{k}$ (as opposed to not having the characteristic) results in an expected change of $\beta_{k}$ in $Y$, holding all other variables constant.}
			\item When there is no interaction term present, $\mbox{Marginal Change}=\mbox{Discrete Change}$
		\end{itemize}
\end{frame}

\begin{frame}
	\frametitle{Marginal Effects}
		\begin{figure}[p]
			\centering
			%\includegraphics[scale=0.3]{Graphs/mfxOLS.png}
			\label{fig:fig1}
		\end{figure}
\end{frame}

\section{\texttt{margins} and \texttt{marginsplot}}
\subsection{\texttt{margins}}

\begin{frame}
	\frametitle{\texttt{margins}}
	\begin{itemize}
		\item Computes predicted values and marginal effects from last estimated regression model
		\item Reports computed statistic, standard error, test statistic, $p$-value and 95\% CI.
		\item \texttt{at(atspec)} option allows for the calculation of predicted values and marginal effects at specific values of independent variable(s).
		\item \texttt{dydx()} option allows for calculating marginal effects.
		\item Factor variables (\texttt{i.\underline{varname}}) can go after the \texttt{margins} command or within the \texttt{at(atspec)} option.
		\item Continuous variables can only be specified within the \texttt{at(atspec)} option.
		\item \texttt{atmeans} option sets variables not specified to be held at their mean value.
	\end{itemize}
\end{frame}

\begin{frame}
	\frametitle{Predicted (Fitted) Values -- \texttt{margins} Syntax}
		\begin{itemize}
			\item \texttt{margins} -- Overall predicted value with all independent variables held at their mean value.
			\item \texttt{margins, at(varname=\#)} -- Predicted value when one or more independent variables are fixed to a specific value and remaining independent variables held at their mean value.
			\item \texttt{margins, at(varname=numlist)} -- Predicted value(s) when one or more independent variables are fixed to multiple values and remaining independent variables held at their mean value.
			\item \texttt{margins varname} -- Overall predicted value(s) for categories of \texttt{varname} with remaining independent variables held at their mean value.
		\end{itemize}
\end{frame}

\begin{frame}
	\frametitle{Marginal Change -- \texttt{margins} Syntax}
		\begin{itemize}
			\item \texttt{margins, dydx(varname)} -- Average marginal effect a one-unit increase in \texttt{varname} has on the dependent variable, holding all other variables constant.
		\end{itemize}
\end{frame}

\begin{frame}
	\frametitle{Discrete Change -- \texttt{margins} Syntax}
		\begin{itemize}
			\item \texttt{margins, at(varname=(\textit{start} \textit{end})) post} -- Calculates predicted values at specified values, and treats results as estimation results.
			\item \texttt{lincom $\mbox{2.\_at}-\mbox{1.\_at}$} -- Calculates the difference between the prediction of the ending value and the prediction of the starting value.
		\end{itemize}
\end{frame}

\subsection{\texttt{marginsplot}}

\begin{frame}
	\frametitle{\texttt{marginsplot}}
	\begin{itemize}
		\item Graphs the results of last estimated \texttt{margins} command
		\item \textbf{Needs to be executed immediately after \texttt{margins}}
		\item Resulting graph includes an overall title, a title for the $y$-axis, $x$-axis features the name of the variable (variable label if one is included).
		\item The featured values on the $x$-axis are the values specified from the \texttt{margins} command.
		\item Can use the \texttt{recast} and \texttt{recastci} options to change how results are graphed.
	\end{itemize} 
\end{frame}

\begin{frame}
	\begin{center}
		\begin{LARGE}
			Email: desmond-wallace@uiowa.edu\\
			Any Questions?
		\end{LARGE}
	\end{center}
\end{frame}

\end{document}
