\documentclass{beamer}
\setbeamertemplate{caption}[numbered]
\usepackage{graphicx}
\usepackage{svg}
\usepackage[utf8]{inputenc}
\usepackage[english]{babel}
\graphicspath{{Graphs/}}
\usepackage{comment}
\usepackage{verbatim}
\usepackage{hyperref}
\mode<presentation>
{
	\usetheme{AnnArbor}
	\usecolortheme{crane}
}

\title[MLM I]{Multilevel Modeling (Part 2)}
\subtitle[ISRC Workshop]{Random-Intercept and Random-Slope Modeling with Covariates}
\author[Wallace]{Desmond D. Wallace}
\institute[University of Iowa]{Department of Political Science\\The University of Iowa\\Iowa City, IA}

\date{April 13, 2018}

\begin{document}

\begin{frame}
	\titlepage
\end{frame}

%\begin{frame}
%	\frametitle{Table of Contents}
%	\tableofcontents
%\end{frame}

\section{Multilevel Modeling}

\subsection{Random Intercept Model}

\begin{frame}
	\frametitle{Including Level-1 Covariates}
		\begin{itemize}
			\item Predicting the outcome from an intercept that varies between groups and individual-level independent variables.
			\item The 2-level model takes the following form:
				\begin{itemize}
					\item Level-1 Model: $y_{ij}=\beta_{0j}+\beta_{1j}x_{ij}+\varepsilon_{ij}$
					\item Level-2 Models
						\begin{itemize}
							\item Intercept: $\beta_{0j}=\gamma_{00}+U_{0j}$ where
								\begin{description}[abc]
									\item $\gamma_{00}$ -- Average (general) intercept holding across all groups (fixed effect)
									\item $U_{0j}$ -- Group-specific effect on the intercept (random effect)
								\end{description}
							\item Slope: $\beta_{1j}=\gamma_{10}$ where
								\begin{description}[abc]
									\item $\gamma_{10}$ -- Amount of increase (decrease) in dependent variable for a one-unit change in $x_{ij}$ (fixed effect)
								\end{description}
						\end{itemize}
					\item Full Specification: $y_{ij}=\gamma_{00}+\gamma_{10}x_{ij}+U_{0j}+\varepsilon_{ij}$
				\end{itemize}
		\end{itemize}
\end{frame}

\begin{frame}
	\frametitle{Assumptions}
		\begin{enumerate}
			\item $U_{0j}$ and $\varepsilon_{ij}$ are mutually independent with mean $0$, given the values of $x_{ij}$
			\item $U_{0j}$ is randomly drawn from a population distribution with mean $0$ and variance $\tau^{2}_{0}$
			\item Population variance of level-1 residuals, $\sigma^2$, is constant across groups
			\item $U_{0j}$ are interpretable as group-level residuals, or group effects left unexplained by $x_{ij}$
			\item \textbf{Unexplained variability at multiple levels is essence of multilevel modeling}
		\end{enumerate}
\end{frame}

\begin{frame}
	\frametitle{Variance of $y_{ij}$}
		\begin{itemize}
			\item Variance of $y_{ij}$, conditioned on the values of $x_{ij}$ is again the sum of the level-two and level-one variances
				\begin{itemize}
					\item $Var(y_{ij}\mid X_{ij})=Var(U_{0j})+Var(\varepsilon_{ij})=\tau^{2}_{0}+\sigma^2$
				\end{itemize}
			\item Covariance between two observations from the same group $(ij\mbox{ and }i'j)$ is equal to the variance of the contribution $U_{0j}$ that is shared by these observations
				\begin{itemize}
					\item $Cov(y_{ij},y_{i'j}\mid x_{ij},x_{i'j})=Var(U_{0j})=\tau^{2}_{0}$
				\end{itemize}
		\end{itemize}
\end{frame}

\begin{frame}
	\frametitle{Residual Intraclass Correlation Coefficient}
		\begin{itemize}
			\item A part of the covariance or correlation between two observations from the same group is \textit{explained} by values of the independent variable(s)
			\item The rest of the covariance or correlation is \textit{unexplained}
			\item $\rho(y\mid X)=\frac{\tau^{2}_{0}}{\tau^{2}_{0}+\sigma^2}$
			\item Represents correlation between the $y$-values of two randomly drawn individuals in a randomly drawn group, controlling for $x$
			\item If $\rho(y\mid X)=0$, OLS is appropriate; If $\rho(y\mid X)>0$, then multilevel model is better
		\end{itemize}
\end{frame}

\begin{frame}
	\frametitle{Example: Math Achievement}
		\begin{itemize}
			\item Goal: Examine the influence students' socioeconomic status (SES) has on math achievement scores while controlling for students' minority and gender identification.
			\item Model Specification
				\begin{itemize}
					\item Level-1 Model (student-level): $\mbox{mathach}_{ij}=\beta_{0j}+\beta_{1j}\mbox{SES}_{ij}+\beta_{2j}\mbox{minority}_{ij}+\beta_{3j}\mbox{female}_{ij}+\varepsilon_{ij}$
					\item Level-2 Models (school-level):
						\begin{itemize}
							\item $\beta_{0j}=\gamma_{00}+U_{0j}$
							\item $\beta_{1j}=\gamma_{10}$
							\item $\beta_{2j}=\gamma_{20}$
							\item $\beta_{3j}=\gamma_{30}$
						\end{itemize}
					\item Full Model: $\mbox{mathach}_{ij}=\gamma_{00}+\gamma_{10}\mbox{SES}_{ij}+\gamma_{20}\mbox{minority}_{ij}+\gamma_{30}\mbox{female}_{ij}+U_{0j}+\varepsilon_{ij}$
				\end{itemize}
		\end{itemize}
\end{frame}

\begin{frame}
	\frametitle{Example: Math Achievement}
		\begin{itemize}
			\item Parameter Interpretation
				\begin{itemize}
					\item $\gamma_{00}$: Overall mean of student's math achievement scores
					\item $\gamma_{10}$: Effect SES has on math achievement
					\item $\gamma_{20}$: Difference in math achievement between minorities and non-minorities
					\item $\gamma_{30}$: Difference in math achievement between females and males
					\item $U_{0j}$: Unique effect of school $j$ on mean math achievement score
				\end{itemize}
		\end{itemize}
\end{frame}

\subsection{Random Slope Model}

\begin{frame}
	\frametitle{Random Slopes}
		\begin{itemize}
			\item Belief the relationship between independent and dependent variables differs across groups
			\item The 2-level model takes the following form:
				\begin{itemize}
					\item Level-1 Model: $y_{ij}=\beta_{0j}+\beta_{1j}x_{ij}+\varepsilon_{ij}$
					\item Level-2 Models
						\begin{itemize}
							\item Intercept: $\beta_{0j}=\gamma_{00}+U_{0j}$ where
								\begin{description}[abc]
									\item $\gamma_{00}$ -- Average (general) intercept holding across all groups (fixed effect)
									\item $U_{0j}$ -- Group-specific effect on the intercept (random effect)
								\end{description}
							\item Slope: $\beta_{1j}=\gamma_{10}+U_{1j}$ where
								\begin{description}[abc]
									\item $\gamma_{10}$ -- Average relationship of $x_{ij}$ and $y_{ij}$ across groups (fixed effect)
									\item $U_{1j}$ -- Group-specific variation of the relationship between $x_{ij}$ and $y_{ij}$ (random effect)
								\end{description}
						\end{itemize}
					\item Full Specification: $y_{ij}=\gamma_{00}+\gamma_{10}x_{ij}+U_{0j}+U_{1j}x_{ij}+\varepsilon_{ij}$
				\end{itemize}
		\end{itemize}
\end{frame}

\begin{frame}
	\frametitle{Assumptions}
		\begin{itemize}
			\item All residuals ($U_{0j}$, $U_{1j}$, and $\varepsilon_{ij}$) have mean $0$, given the values of the independent variable(s)
			\item The pair of random effects ($U_{0j}$, $U_{1j}$) are independent and identically distributed ($i.i.d$)
			\item ($U_{0j}$, $U_{1j}$) are independent of ($\varepsilon_{ij}$)
			\item $\varepsilon_{ij}$ is $i.i.d$
		\end{itemize}
\end{frame}

\begin{frame}
	\frametitle{Variances}
		\begin{itemize}
			\item Random Effects
				\begin{itemize}
					\item $Var(U_{0j})=\tau_{00}=\tau_{0}^{2}$
					\item $Var(U_{1j})=\tau_{11}=\tau_{1}^{2}$
					\item $Cov(U_{0j},U_{1j})=\tau_{01}$
				\end{itemize}
			\item $y_{ij}$
				\begin{itemize}
					\item $Var(y_{ij}\mid X_{ij})=\tau_{0}^{2}+2\tau_{01}x_{ij}+\tau_{1}^{2}x_{ij}^{2}+\sigma^{2}$
					\item $Cov(y_{ij},y_{i'j}\mid x_{ij},x_{i'j})=\tau^{2}_{0}+\tau_{01}(x_{ij}+x_{i'j})+\tau_{1}^{2}x_{ij}x_{i'j}$
					\item Residual variance is minimal for $x_{ij}=\frac{-\tau_{01}}{\tau_{1}^{2}}$
						\begin{itemize}
							\item If within the range of possible $x_{ij}$ values, variance will first decrease, then increase
							\item If smaller than all $x_{ij}$ values, variance will increase as a function of $x$
							\item If larger than all $x_{ij}$ values, variance will decrease as a function of $x$
						\end{itemize}
				\end{itemize}
		\end{itemize}
\end{frame}

\begin{frame}
	\frametitle{Example: Math Achievement}
		\begin{itemize}
			\item Goal: Examine the influence students' socioeconomic status (SES) has on math achievement scores while controlling for students' minority and gender identification, while accounting for the effect of SES varying across schools.
			\item Model Specification
				\begin{itemize}
					\item Level-1 Model (student-level): $\mbox{mathach}_{ij}=\beta_{0j}+\beta_{1j}\mbox{SES}_{ij}+\beta_{2j}\mbox{minority}_{ij}+\beta_{3j}\mbox{female}_{ij}+\varepsilon_{ij}$
					\item Level-2 Models (school-level):
						\begin{itemize}
							\item $\beta_{0j}=\gamma_{00}+U_{0j}$
							\item $\beta_{1j}=\gamma_{10}+U_{1j}$
							\item $\beta_{2j}=\gamma_{20}$
							\item $\beta_{3j}=\gamma_{30}$
						\end{itemize}
					\item Full Model: $\mbox{mathach}_{ij}=\gamma_{00}+\gamma_{10}\mbox{SES}_{ij}+\gamma_{20}\mbox{minority}_{ij}+\gamma_{30}\mbox{female}_{ij}+U_{0j}+U_{1j}+\varepsilon_{ij}$
				\end{itemize}
		\end{itemize}
\end{frame}

\begin{frame}
	\frametitle{Example: Math Achievement}
		\begin{itemize}
			\item Parameter Interpretation
				\begin{itemize}
					\item $\gamma_{00}$: Overall mean of student's math achievement scores
					\item $\gamma_{10}$: Average effect SES has on math achievement
					\item $\gamma_{20}$: Difference in math achievement between minorities and non-minorities
					\item $\gamma_{30}$: Difference in math achievement between females and males
					\item $U_{0j}$: Unique effect of school $j$ on mean math achievement score
					\item $U_{1j}$: Unique effect of school $j$ on SES effect on math achievement score
				\end{itemize}
		\end{itemize}
\end{frame}

\subsection{Model Extensions}

\begin{frame}
	\frametitle{Explaining Random Intercept and Random Slope Variation}
		\begin{itemize}
			\item So far, coefficients have been the sum of an average and random effect.
			\item One could further explain this random variability via inclusion of group-level variables ($Z$)
			\item Example (Single group-level variable):
				\begin{itemize}
					\item Random Intercept: $\beta_{0j}=\gamma_{00}+\gamma_{01}z_{j}+U_{0j}$
					\item Random Slope: $\beta_{1j}=\gamma_{10}+\gamma_{11}z_{j}+U_{1j}$
				\end{itemize}
			\item Including a group-level variable in the random intercept equation leads to a main effect of $z_{j}$
			\item Including a group-level variable in the random slope equation leads to an interaction effect of $z_{j}x_{ij}$ (Cross-level Interaction)
			\item Just as with level-1 variables, can feature multiple level-2 variables
		\end{itemize}
\end{frame}

\begin{frame}
	\frametitle{Example: Math Achievement}
		\begin{itemize}
			\item Goal: Examine the influence students' socioeconomic status (SES) has on math achievement scores.
			\item Model Specification
				\begin{itemize}
					\item Level-1 Model (student-level): $\mbox{mathach}_{ij}=\beta_{0j}+\beta_{1j}\mbox{SES}_{ij}+\beta_{2j}\mbox{minority}_{ij}+\beta_{3j}\mbox{female}_{ij}+\varepsilon_{ij}$
					\item Level-2 Models (school-level):
						\begin{itemize}
							\item $\beta_{0j}=\gamma_{00}+\gamma_{01}\mbox{size}_{j}+\gamma_{02}\mbox{sector}_{j}+U_{0j}$
							\item $\beta_{1j}=\gamma_{10}+\gamma_{11}\mbox{size}_{j}+U_{1j}$
							\item $\beta_{2j}=\gamma_{20}$
							\item $\beta_{3j}=\gamma_{30}$
						\end{itemize}
					\item Full Model: $\mbox{mathach}_{ij}=\gamma_{00}+\gamma_{01}\mbox{size}_{j}+\gamma_{02}\mbox{sector}_{j}+\gamma_{10}\mbox{SES}_{ij}+\gamma_{11}\mbox{size}_{j}*\mbox{SES}_{ij}+\gamma_{20}\mbox{minority}_{ij}+\gamma_{30}\mbox{female}_{ij}+U_{0j}+U_{1j}+\varepsilon_{ij}$
				\end{itemize}
		\end{itemize}
\end{frame}

\begin{frame}
	\frametitle{Example: Math Achievement}
		\begin{itemize}
			\item Parameter Interpretation
				\begin{itemize}
					\item $\gamma_{00}$: Overall mean of student's math achievement scores
					\item $\gamma_{01}$: Effect school size has on overall mean of student's math achievement scores when $\mbox{SES}=0$
					\item $\gamma_{02}$: Difference in overall mean of student's math achievement scores for schools in sectors coded as $1$ compared to schools in sectors coded as $0$.
					\item $\gamma_{10}$: Average effect SES has on math achievement when $\mbox{size}=0$
					\item $\gamma_{11}$: Average effect SES has on math achievement depends on school size
					\item $\gamma_{20}$: Difference in math achievement between minorities and non-minorities
					\item $\gamma_{30}$: Difference in math achievement between females and males
					\item $U_{0j}$: Unique effect of school $j$ on mean math achievement score
					\item $U_{1j}$: Unique effect of school $j$ on SES effect on math achievement score
				\end{itemize}
		\end{itemize}
\end{frame}

\begin{frame}
	\begin{center}
		\begin{LARGE}
			Email: desmond-wallace@uiowa.edu\\
			Any Questions?
		\end{LARGE}
	\end{center}
\end{frame}

\end{document}
